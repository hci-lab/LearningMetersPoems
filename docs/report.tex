\documentclass[12pt]{article}
\usepackage{multicol}


%% Drawing
\usepackage{tikz}
\usetikzlibrary{arrows} % For arrows :"D

% For Arabic
\usepackage{polyglossia}
\setmainlanguage{english}
\setotherlanguage{arabic}
\newfontfamily\arabicfont[Script=Arabic]{Amiri}



\begin{document}


\begin{titlepage}

\newcommand{\HRule}{\rule{\linewidth}{0.5mm}} % Defines a new command for the horizontal lines, change thickness here

\center % Center everything on the page
 
%-----------------------
%	HEADING SECTIONS
%-----------------------

\textsc{\LARGE Helwan University}\\[1.5cm] % Name of your university/college
\textsc{\Large \textarabic{تعرفهم على الأنماط حائر}}\\[0.5cm] % Major heading such as course name

%-----------------------
%	TITLE SECTION
%-----------------------

\HRule \\[0.4cm]
{ \LARGE \bfseries Learning meters of Arabic and English poems}\\[0.4cm] % Title of your document
\HRule \\[1.5cm]
 
%-----------------------
%	AUTHOR SECTION
%-----------------------

\begin{minipage}{0.4\textwidth}
\begin{flushleft} \large
\emph{Members, \textit{\small alphabetically ordered}:}\\

\small{Abdaullah Ramzy}\\
\small{Ali Abdemoniem}\\
\small{Ali Osama}\\
\small{Taha Magdy}\\
\small{Umar Mohamed}\\
\end{flushleft}
\end{minipage}
~
\begin{minipage}{0.4\textwidth}
\begin{flushright} \large
\emph{Supervisor:} \\
Prof. Waleed A.\textsc{Yousuf} % Supervisor's Name
\end{flushright}
\end{minipage}\\[2cm]

% If you don't want a supervisor, uncomment the two lines below and remove the section above
%\Large \emph{Author:}\\
%John \textsc{Smith}\\[3cm] % Your name

%-----------------------
%	DATE SECTION
%-----------------------

{\large \today}\\[2cm] % Date, change the \today to a set date if you want to be precise

%-----------------------
%	LOGO SECTION
%-----------------------
\includegraphics[width=30mm,scale=0.5]{logo.png}\\[1cm] % Include a department/university logo - this will require the graphicx package

\vfill % Fill the rest of the page with whitespace

\end{titlepage}
%\tableofcontents \newpage




%%%%%%
\section{Introduction and Problem Statement}
Detecting the meter of poems is not an easy task for ordinary people, but how
computers will perform? Our task is to train a model so that it can detect the
meter of the input verse/text.
We have worked on Arabic and English in parallel, everything thing is applied to
Arabic is applied also in English, as possible as we can.

To be clearer, the model's input is a verse/text \textarabic{بيت شعر} and the
output is a class which is the verse's meter \textarabic{البحر}, as shown in the
figure below.


% FIGURE
\begin{center}
\begin{tikzpicture}
\centering

%\draw[step=0.5, gray, very thin] (0,0) grid (6,2);
\node at (3, 1.5) {Deep};
\draw[rounded corners=2pt, thick] (2,0) rectangle (2+2,2) node[pos=.5] {Learning};
\node at (3, 0.5) {Model};

\draw[arrows=-angle 90, line width=1pt ] (1, 1) -- (1 +1 -0.1, 1);
\node at (0, 1.5) {Verse};
\node at (0, 1) {\textarabic{بيتُ الشعر}};

\draw[arrows=-angle 90, line width=1pt] (4 +.1, 1) -- (4 +1, 1);
\node at (6, 1.5) {Meter};
\node at (6, 1) {\textarabic{البحْرُ}};

\end{tikzpicture}
\end{center}

The output is a class, then our problem can be described as \textit{supervised
learning  classification}.  We have trained some deep learning models such as
LSTM, Bi-LSTM and GRU.  Those models are chosen because of the nature of our
problem. We were trying to detect the verse's meter, which is a sequence of
characters and \textit{recurrent neural network} are suitable  to learn that
pattern, thanks to its cell's share-memory and its recursive structure.


\section{The Project Road Map}

% FIGURE
\begin{center}
\begin{tikzpicture}
\centering

\draw[step=0.5, gray, very thin] (-6,0) grid (6,2);

\end{tikzpicture}
\end{center}

\section{Objectives}
\section{Tools}
Python is pseudo-code like programming language, it is so easy and high-level
that we can describe complex structures in a few lines of code, the main second
reason is that python recently has been so papular in the Artificial Intelligence
community. Its library is so rich with packages for Machine Learning, Deep
Learning, data manipulation, even for web-scraping; we don't need to parse HTML
by you hands.

We have used:
Two columns:
    \begin{multicols}{2}
\begin{itemize}
\item \textit{Python} 3.6.5
\item \textit{Keras} x.x for deep learning.
\item \textit{Tensorflow} x.x as back-end of Keras.
\item \textit{BeautifulSoup} for web scraping.
\end{itemize}
    \end{multicols}





\section{Gathering data}

% Figure of Project flow

\end{document}


