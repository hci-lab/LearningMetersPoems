\documentclass[10pt]{beamer}

\usetheme[progressbar=frametitle]{metropolis}
\usepackage{appendixnumberbeamer}
\usepackage{booktabs}
\usepackage[scale=2]{ccicons}
\usepackage{pgfplots}
\usepgfplotslibrary{dateplot}
\usepackage{xspace}
\usepackage{amsmath}
\newcommand{\themename}{\textbf{\textsc{metropolis}}\xspace}

\usepackage{tipa}
\usepackage{xcolor}

\definecolor{myBlue}{HTML}{7982db}
\definecolor{myGreen}{HTML}{ABDDA4}
\definecolor{myBlue2}{HTML}{4169AA}
\definecolor{myGreen2}{HTML}{38AE40}




% For Arabic
\usepackage{polyglossia} 
\setmainlanguage{english}
\setotherlanguage{arabic}
% Defining the font family
\newfontfamily\arabicfont  [Script=Arabic, Scale=1.5]{Scheherazade}%
\newfontfamily\arabicfontsf[Script=Arabic, Scale=1.5]{Scheherazade}%

\usepackage{bidipoem}

\title{Learning Meters of Arabic and English poems}
\subtitle{With Recurrent Neural Networks}
\date{\today}
\author{
Prof. Waleed A. YOUSEF \\
The Team}
\institute{Computer Science department\\
Faulty of Computers and Information, Helwan University}

%\title{A Practical Introduction to Natural Language Processing}
%
%\subtitle{Intelligent Processing \& Applications\\Research Cluster Seminar}
%
%%\date{5 \& 12 March 2015}
%%\date{5 March 2015\\Session 1: Common Tasks and Concepts in NLP}
%%\date{12 March 2015\\Session 2: Software Libraries and Resources for NLP}
%\date{5 March 2015\\Session 1: Common Tasks and Concepts in NLP\\[0.5ex]
%12 March 2015\\Session 2: Software Libraries and Resources for NLP}
%\author{Dr Lim Lian Tze}
%\institute{
%Information Technology Department\\
%School of Science, Engineering and Technology\\
%KDU College Penang
%}

\begin{document}

\maketitle

\begin{frame}{Table of contents}
  \setbeamertemplate{section in toc}[sections numbered]
  \tableofcontents[hideallsubsections]
\end{frame}



% 1
\section{Introduction}
\begin{frame}[fragile]{Hello, Arabic}
    \begin{center}
    \begin{Arabic}
    ودعْ عنك آراءَ الرجالِ وقولَهم\hspace{3em}  فقولُ رسولِ الله أزكى وأشرحُ
    \end{Arabic}
    \end{center}
\end{frame}


% But ... What is poetry?
\begin{frame}[fragile]{But ... What is poetry?}

\textbf{\large General Definition}:
\begin{itemize}
  \item \textbf{Poetry} is a piece of writing or speaking, which
\textbf{\textcolor{red}{MUST}} follow specific
\alert{\underline{\textbf{Patterns}}}.
\end{itemize}

\vspace{0.5cm}
\textbf{\large Example}, \textit{\small English verse}:
\begin{center}
That 
  \textcolor{myGreen2}{\textbf{time}} of 
  \textcolor{myGreen2}{\textbf{year}} thou
  \textcolor{myGreen2}{\textbf{mayst}}  in 
  \textcolor{myGreen2}{\textbf{me}}
be\textcolor{myGreen2}{\textbf{hold}}
\end{center}

To detect poems' meters, we need to learn those \alert{\textbf{Patterns}}.
\end{frame}




\begin{frame}[fragile]{Arabic Prosody \textarabic{العَرُوض}}

\begin{itemize}
  \item \textbf{Foot} \textarabic{التفعيلة}: is a sequence of vowels and consonants.
\end{itemize}




\begin{center}
  \begin{tabular}{|c|c|} \hline
    \textbf{Feet} & \textbf{Scansion} \\
    \hline
    \textarabic{فَعُولُنْ}  & \texttt{0/0//}\\
    \textarabic{فَاعِلُنْ}  & \texttt{0//0/}\\
    \textarabic{مُسْتَفْعِلُنْ}& \texttt{0//0/0/}\\
    \textarabic{مَفاعِيلُنْ}& \texttt{0/0/0//}\\
    \textarabic{مَفْعُولاَت} & \texttt{0//0///}\\
    \textarabic{فَاعِلاَتُنْ} & \texttt{0/0//0/}\\
    \textarabic{مُفَاعَلَتُنْ}& \texttt{0///0//}\\
    \textarabic{مُتَفَاعِلُنْ}& \texttt{0//0///}\\
    \hline
  \end{tabular}
\end{center}

\end{frame}





\begin{frame}[fragile]{Arabic Prosody \textarabic{العَرُوض}}

Arabic Patterns/Meters \textarabic{بحور الشعر}:
\begin{itemize}
  \item \textbf{Meter} \textarabic{البحر}: is a sequence of \alert{feet}. 
\end{itemize}

\begin{center}
  \begin{tabular}[h!]{|c|c|} 
    \hline
    \textbf{Meter Name} & \textbf{Meter} \small{\textit{feet combination}} \\ 
    \hline
   \textit{al-Wafeer}    & \textarabic{مُفَاعَلَتُن مُفَاعَلَتُن فَعُولُن} \\ %
   \textit{al-Taweel}    & \textarabic{فَعُوْلُنْ مَفَاْعِيْلُنْ فَعُوْلُنْ مَفَاْعِلُنْ} \\ %
   \vdots                &  \vdots\\
   \textit{al-Moktadib}  & \textarabic{مَفْعُوْلاتُ مُسْتَفْعِلُنْ مُسْتَفْعِلُن} \\
   \textit{al-Modar'e}   & \textarabic{مَفَاْعِيْلُنْ فَاْعِلاتُنْ مَفَاْعِيْلُنْ} \\
    \hline
  \end{tabular}
\end{center}

\end{frame}




\begin{frame}[fragile]{Arabic Prosody, example!}

\textbf{From \textarabic{بحر الوافر}}:
\begin{center}
  \textarabic{ويسْأل فىْ الْحواْدث ذوْ صواْبٍ}\\
  \textarabic{ويسأل فل \hspace{0.4cm}
    حوادث ذو\hspace{0.4cm}
    صوابن}\\
  \texttt{0/0//     \hspace{0.3cm}
          0///0//   \hspace{0.3cm}
          0///0//}\\

  \textarabic{مفاْعلتنْ\hspace{0.7cm} 
    مفاْعلتنْ          \hspace{0.7cm}
    فعوْلنْ}
\end{center}

\end{frame}





\begin{frame}[fragile]{English Prosody}


\textbf{English Meters Building Blocks}:
\begin{itemize}
  \item Syllables: \textipa{\sffamily /"wO:t@/} = \textipa{\sffamily /"wO:/}  $+$ \textipa{\sffamily /t@(r)/}.
    \begin{itemize}
         \item \textcolor{myGreen2}{\textbf{stressed}} $+$ unstressed.
    \end{itemize}
  \item Foot: is a combination of stressed and unstressed syllables. 

\end{itemize}

\begin{center}
\begin{tabular}{|c | c|} 
    \hline
    %\toprule
    Feet     & Stresses Combination\\ 
    \hline
    %\toprule
    \textit{Iamb} & $\times$\textit{/}\\             %\midrule
    \textit{Trochee}& \textit{/}$\times$\\           %\midrule
    \textit{Dactyl} & \textit{/}$\times\times$\\     %\midrule
    \textit{Anapest}& $\times\times$\textit{/}\\     %\midrule
    \textit{Pyrrhic}& $\times\times$\\               %\midrule
    \textit{Amphibrach}& $\times$\textit{/}$\times$\\%\midrule
    \textit{Spondee}& \textit{/}\textit{/}\\
    %\bottomrule
    \hline
\end{tabular}
\end{center}

\textbf{Meter}: is repeating a foot $n$ times; where $n \in [1, 8]$. 
\end{frame}





\begin{frame}[fragile]{English Patterns}

Iambic pentameter verse:
\begin{center}
$\underbrace{\text{That \textcolor{myGreen2}{\textbf{time}}}
}_\text{Iambic Foot}$
%
$\overbrace{\text{of \textcolor{myGreen2}{\textbf{year}}}
}^\text{2nd}$
%
$\overbrace{\text{thou \textcolor{myGreen2}{\textbf{mayst}}}
}^\text{3rd}$
%
$\overbrace{\text{in \textcolor{myGreen2}{\textbf{me}}}
}^\text{4th}$
%
$\overbrace{\text{be\textcolor{myGreen2}{\textbf{hold}}}
}^\text{5th}$
\end{center}
\end{frame}


%\begin{frame}[fragile]{Arabic Patterns}
%Hello
%\end{frame}










\section{Literature Review}
\section{Technical Details}
\section{Results}  









\end{document}
